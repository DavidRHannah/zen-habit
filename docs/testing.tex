\documentclass[11pt]{article}
\usepackage[utf8]{inputenc}
\usepackage[T1]{fontenc}
\usepackage{lmodern}
\usepackage{hyperref}
\usepackage{geometry}
\usepackage{listings}
\usepackage{xcolor}
\usepackage{enumitem}
\usepackage{parskip}
\geometry{margin=1in}

\hypersetup{
  colorlinks=true,
  linkcolor=blue,
  urlcolor=blue,
  pdftitle={My Study Tool - Testing and Deployment Documentation},
  pdfpagemode=FullScreen,
}

\definecolor{codegray}{rgb}{0.5,0.5,0.5}
\definecolor{codepurple}{rgb}{0.58,0,0.82}
\definecolor{backcolour}{rgb}{0.95,0.95,0.92}

\lstdefinestyle{mystyle}{
    backgroundcolor=\color{backcolour},   
    commentstyle=\color{codegray},
    keywordstyle=\color{blue},
    numberstyle=\tiny\color{codegray},
    stringstyle=\color{codepurple},
    basicstyle=\footnotesize\ttfamily,
    breakatwhitespace=false,         
    breaklines=true,                 
    captionpos=b,                    
    keepspaces=true,                 
    numbers=left,                    
    numbersep=5pt,                  
    showspaces=false,                
    showstringspaces=false,
    showtabs=false,                  
    tabsize=2
}

\lstset{style=mystyle}

\title{My Study Tool - Testing and Deployment Documentation}
\author{Your Name}
\date{\today}

\begin{document}

\maketitle
\tableofcontents
\newpage

\section{Introduction}
This document provides a step-by-step guide for testing, debugging, and deploying the My Study Tool application. It covers local development setup, using Stripe webhooks with the Stripe CLI, and deploying to a production environment (e.g., Vercel).

\section{Local Development Setup}

\subsection{Clone the Repository}
Open your terminal and run:
\begin{lstlisting}[language=bash]
git clone https://github.com/yourusername/my-study-tool.git
cd my-study-tool
\end{lstlisting}

\subsection{Install Dependencies}
Run the following command:
\begin{lstlisting}[language=bash]
npm install
\end{lstlisting}
Alternatively, if you use Yarn:
\begin{lstlisting}[language=bash]
yarn install
\end{lstlisting}

\subsection{Configure Environment Variables}
Create a file named \texttt{.env.local} in the root of your project and add the following content:
\begin{lstlisting}[language=bash]
NEXT_PUBLIC_SUPABASE_URL=https://your-supabase-project-url.supabase.co
NEXT_PUBLIC_SUPABASE_ANON_KEY=your_supabase_anon_key
STRIPE_SECRET_KEY=your_stripe_secret_key
STRIPE_PRICE_ID=your_stripe_price_id
STRIPE_WEBHOOK_SECRET=your_stripe_webhook_secret
NEXT_PUBLIC_BASE_URL=http://localhost:3000
NEXT_PUBLIC_STRIPE_PUBLISHABLE_KEY=your_stripe_publishable_key
\end{lstlisting}
Adjust the values as needed.

\subsection{Database Setup}
Ensure your Supabase project contains the required tables:
\begin{itemize}[leftmargin=*]
    \item \textbf{Profiles:} Stores extra user details such as \texttt{stripe\_customer\_id} and \texttt{subscription\_status}.
    \item \textbf{Habits:} Tracks user habits, completion status, and points.
    \item \textbf{Study Sessions:} Logs study sessions and earned points.
\end{itemize}

\section{Local Testing}

\subsection{Running the Development Server}
Start your Next.js app in development mode:
\begin{lstlisting}[language=bash]
npm run dev
\end{lstlisting}
Then, visit \url{http://localhost:3000} in your browser to test pages such as Home, Dashboard, and Subscription.

\subsection{Testing Supabase Integration}
\begin{itemize}[leftmargin=*]
    \item \textbf{User Data \& Tables:} Use the Supabase dashboard to verify that your tables (\texttt{profiles}, \texttt{habits}, \texttt{study\_sessions}) are correctly receiving data when you create or update records from your application.
    \item \textbf{Real-Time Updates:} If you use Supabase's real-time features, check that changes are immediately reflected in your application UI.
\end{itemize}

\subsection{Testing Stripe Webhooks}

\subsubsection{Using the Stripe CLI}

\paragraph{Install the Stripe CLI:}
Follow the installation guide available at \url{https://stripe.com/docs/stripe-cli#install}. For example, on macOS:
\begin{lstlisting}[language=bash]
brew install stripe/stripe-cli/stripe
\end{lstlisting}

\paragraph{Login and Authenticate:}
Run:
\begin{lstlisting}[language=bash]
stripe login
\end{lstlisting}

\paragraph{Forward Webhook Events:}
Run the following command to forward Stripe events to your local webhook endpoint:
\begin{lstlisting}[language=bash]
stripe listen --forward-to http://localhost:3000/api/stripe-webhook
\end{lstlisting}
The CLI will output a webhook secret; verify that this secret matches your \texttt{STRIPE\_WEBHOOK\_SECRET} in your \texttt{.env.local}.

\paragraph{Trigger Test Events:}
To simulate a checkout session completion, run:
\begin{lstlisting}[language=bash]
stripe trigger checkout.session.completed
\end{lstlisting}
Check your terminal and logs to ensure that the event is received and processed by your webhook endpoint (e.g., updating the Supabase \texttt{profiles} table).

\section{Automated Testing}

\subsection{Unit and Integration Tests}
Consider setting up tests using frameworks such as \href{https://jestjs.io/}{Jest} and \href{https://testing-library.com/docs/react-testing-library/intro}{React Testing Library}. Example commands:
\begin{lstlisting}[language=bash]
npm run test
\end{lstlisting}
or
\begin{lstlisting}[language=bash]
yarn test
\end{lstlisting}

\subsection{Linting}
Ensure your code adheres to style guidelines:
\begin{lstlisting}[language=bash]
npm run lint
\end{lstlisting}
or
\begin{lstlisting}[language=bash]
yarn lint
\end{lstlisting}

\textit{Note: You may need to add specific test configurations and scripts based on your chosen testing framework.}

\section{Deployment Guide}

\subsection{Pre-deployment Checklist}
Before deploying your application, ensure:
\begin{itemize}[leftmargin=*]
    \item All environment variables are set correctly for production (use live API keys for Stripe and the production URL for Supabase).
    \item Your Supabase tables are correctly configured.
    \item The application has been thoroughly tested locally.
    \item You have set up your webhook endpoint in Stripe to point to your production URL (e.g., \url{https://yourdomain.com/api/stripe-webhook}).
\end{itemize}

\subsection{Deploying with Vercel}

\subsubsection{Push Your Code to GitHub (or your Git provider)}
Ensure your latest changes are committed and pushed.

\subsubsection{Sign Up / Log In to Vercel}
Go to \url{https://vercel.com} and link your GitHub repository.

\subsubsection{Configure Environment Variables on Vercel}
In your project dashboard on Vercel, navigate to \textbf{Settings > Environment Variables} and add:
\begin{lstlisting}[language=bash]
NEXT_PUBLIC_SUPABASE_URL
NEXT_PUBLIC_SUPABASE_ANON_KEY
STRIPE_SECRET_KEY
STRIPE_PRICE_ID
STRIPE_WEBHOOK_SECRET
NEXT_PUBLIC_BASE_URL   % Set to your production URL
NEXT_PUBLIC_STRIPE_PUBLISHABLE_KEY
\end{lstlisting}

\subsubsection{Deploy Your Application}
Vercel automatically deploys when you push to the main branch. You can also trigger manual deployments from the Vercel dashboard.

\subsubsection{Verify Deployment}
\begin{itemize}[leftmargin=*]
    \item Visit your production URL.
    \item Test key features (e.g., creating checkout sessions, webhook events, and Supabase integration).
\end{itemize}

\section{Troubleshooting and Next Steps}

\subsection{Logs and Monitoring}
\begin{itemize}[leftmargin=*]
    \item Use Vercel’s dashboard for real-time logs.
    \item Monitor the Stripe Dashboard and Supabase logs for any errors.
\end{itemize}

\subsection{Local Debugging}
\begin{itemize}[leftmargin=*]
    \item Continue using the Stripe CLI to simulate webhook events.
    \item Utilize browser developer tools and your IDE’s debugging features for frontend issues.
\end{itemize}

\subsection{Documentation Updates}
\begin{itemize}[leftmargin=*]
    \item Update this document as you add features or change configurations.
    \item Maintain version control for this documentation alongside your code.
\end{itemize}

\subsection{Continuous Integration}
Consider setting up CI/CD (e.g., using GitHub Actions) to run tests automatically on each push.

\section{Further References}
For additional details, please refer to:
\begin{itemize}[leftmargin=*]
    \item \href{https://stripe.com/docs}{Stripe Documentation}
    \item \href{https://supabase.com/docs}{Supabase Documentation}
    \item \href{https://nextjs.org/docs}{Next.js Documentation}
    \item \href{https://vercel.com/docs}{Vercel Documentation}
\end{itemize}

\end{document}
